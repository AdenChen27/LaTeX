\documentclass[10pt,letterpaper]{amsart}
\usepackage[workingpaper]{adenc}
% options: color, plain, hideproofs, hidemarkings, workingpaper

\title{Sample Document using \texttt{adenc.sty}}
\author{Aden Chen}
\date{September 8th, 2024}



\begin{document}

\maketitle

\tableofcontents

\begin{mdframed}
Check out the \href{https://github.com/AdenChen27/latex}{Github Repository} for \texttt{adenc.sty}. 
\end{mdframed}


\section{Package Options}
The following package options are supported:
\begin{itemize}
  \item \verb|color| adds background colors for theorem environments (see \Cref{sec:thm-env}). 
  \item \verb|plain| uses the default theorem environments (\texttt{definition}, \texttt{plain}, and \texttt{remark}).
  \item \verb|hideproofs| and \verb|hidemarkings| hide, respectively, proof environments and markings generated using the \verb|\markabove| and \verb|\markbelow| commands (see \Cref{sec:marking}). 
  \item \verb|workingpaper| adds a watermark with date on the first page to indicate that the current document is a draft; adds more space to the margin so that notes written with the \verb|\todo| command (using the \verb|todonotes| package) can fully display. 
\end{itemize}

To pass an option, use: \verb|\usepackage[Option]{adenc}| . 

To pass multiple options, use: \verb|\usepackage[Option1, Option2, ...]{adenc}|. 


\section{Theorem Environments}
\label{sec:thm-env}

\begin{definition}
  A definitive \vocab{definition} is a definition, by definition. 
\end{definition}

\begin{lemma}
  A lamentable lemma. 
\end{lemma}

\begin{theorem}
  A towering theorem. 
\end{theorem}

\begin{corollary}
  A cool corollary. 
\end{corollary}

\begin{remark}
  A remarkable remark. 
\end{remark}

\begin{example}
  An exemplary example. 
\end{example}

\begin{problem}
  A problematic problem. 
\end{problem}

\begin{proof}
  A precise proof.
\end{proof}

Numbering can be turned off by using the corresponding \verb|*| versions of the environments (e.g. \verb|theorem*| instead of \verb|theorem|). 

\section{Marking the Document}
\label{sec:marking}
The \verb|\todo| command in the \verb|todonotes| package is a great way to add notes to a document, but among other things, it does not support display style math and, when used frequently, the places to which they point can\todo{For example, this points to the word ``can''. } be hard\todo{And this points to ``hard''. } to decipher. 
It is for these reasons that the following commands are introduced:
\begin{enumerate}[label=(\alph*)]
  \item \verb|\itodo| (inline todo) produces an inline block of notes. 
    This can be used as placeholders for contents to be added later. 
    \itodo{
      This is an example of what notes produced by the \texttt{\textbackslash itodo} command looks like. 
        Unlike the \texttt{\textbackslash todo} command, \texttt{\textbackslash itodo} supports display math:
      \[
        \sum_{n=1}^{\infty} a_n z^n . 
      \] 
    }
  \item \verb|\markabove| and \verb|\markbelow| provide a way to mark texts without altering the spacing. 
  Both commands take two arguments: (1) align method (\verb|l|, \verb|c|, or \verb|r|); and (2) text to display. 
  For example, 

  \begin{mdframed}
    \noindent
    Test test test\markabove{l}{test1} test\markbelow{c}{I'm marking below here}.

    \noindent
    Test test test test.

    \noindent
    Test test test\markabove{c}{above!} test\markbelow{r}{math! \(\alpha\)}.
  \end{mdframed}
  is produced by the following code:
\end{enumerate}
\begin{verbatim}
Test test test\markabove{l}{test1} test\markbelow{c}{I'm marking below here}.

Test test test test.

Test test test\markabove{c}{above!} test\markbelow{r}{math! \(\alpha\)}.
\end{verbatim}




\section{New Commands}
Some new commands (mainly for math symbols) are added or modified for aesthetics and/or convenience.
A few notable ones are mentioned below:

\begin{itemize}
  \item A vocab command for styling new vocabulary (in, for example, definitions): \vocab{the vocab command} (\verb|\vocab{the vocab command}|). 
  \item A contradiction symbol: \(\contradiction\) (\verb|\contradiction|). 
  \item Short cuts for \verb|\mathbb|, \verb|\mathcal|, and \verb|\mathscr|:
    \begin{itemize}
      \item \verb|\cX| produces \verb|\mathbb{X}|, 
      \item \verb|\cX| produces \verb|\mathcal{X}|, 
      \item \verb|\sX| produces \verb|\mathscr{X}|. 
    \end{itemize}
    E.g. \(\RR\) (\verb|\RR|), \(\cT\) (\verb|\cT|), \(\sK\) (\verb|\sK|). 
    Note that these shortcuts are not available for all letters. 
  % \item A better looking mod: \(x \equiv y \mod 3\) (\verb|x \equiv y \mod 3|). 
  \item A better looking complement symbol: \(A^\complement\) (\verb|A^\complement|).
  \item A better empty set symbol: \(\emptyset\) (\verb|\emptyset|). 
  \item A command for vectors: \(\vec{v}\) (\verb|\vec{v}|).
  \item Short cuts for matrices: \[
    \bmat{1 & 2 \\ 3 & 4}, \pmat{1 & 2 \\ 3 & 4}, \vmat{1 & 2 \\ 3 & 4}
  \]
  (\verb|\bmat{1 & 2 \\ 3 & 4}, \pmat{1 & 2 \\ 3 & 4}, \vmat{1 & 2 \\ 3 & 4}|). 
  \item Differentiation operator: \(\d x\) (\verb|\d x|).
  \item Imaginary number: \(\I\) (\verb|\I|).
\end{itemize}



\section{Credits}
I have stolen a lot of stuff from \href{https://web.stanford.edu/~lindrew/}{Andrew Lin}'s package, \href{https://web.stanford.edu/~lindrew/lindrew.sty}{lindrew}, and \href{https://github.com/gillescastel}{Gilles Castel}'s \href{https://github.com/gillescastel/lecture-notes/blob/master/algebraic-topology/preamble.tex}{preamble file} for his \href{https://github.com/gillescastel/lecture-notes}{lecture notes}. 









\end{document}







