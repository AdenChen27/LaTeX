\documentclass[10pt,letterpaper]{amsart}
\usepackage{../adenc}
% \usepackage[plain]{../adenc}
% \usepackage[nocolor]{../adenc}

\title{Sample Document}
\author{Aden Chen}
% \date{July 2024}

\begin{document}

\maketitle

\tableofcontents

\noindent
\textbf{Check out the \href{https://github.com/AdenChen27/latex}{github repo}. }

\section{Theorem Environments}

\begin{definition}
  A \vocab{definition} is a definition, by definition. 
\end{definition}

\begin{lemma}
  Cool lemma. 
\end{lemma}

\begin{theorem}
  Cool theorem. 
\end{theorem}

\begin{corollary}
  Cool corollary. 
\end{corollary}

\begin{remark}
  A quite remarkable remark. 
\end{remark}

\begin{example}
  Nice example. 
\end{example}

\begin{problem}
  Cool problem. 
\end{problem}

\begin{proof}
  Nice proof.
\end{proof}

\begin{itemize}
  \item Numbering can be turned off by using the corresponding \verb|*| versions of the environments (e.g. \verb|theorem*| instead of \verb|theorem|). 
  \item Use \verb|\usepackage[nocolor]{adenc}| to produce only black and white theorem environments; use \verb|\usepackage[plain]{adenc}|  to use the default theorem environments: \texttt{definition}, \texttt{plain}, and \texttt{remark}.
\end{itemize}


\section{Features}

\subsection{General math symbols}
\begin{itemize}
  \item A vocab command for styling new vocabulary (in, for example, definitions): \vocab{the vocab command} (\verb|\vocab{the vocab command}|). 
  \item A contradiction symbol: \(\contradiction\) (\verb|\contradiction|). 
  \item Short cuts for \verb|\mathbb| (\verb|\XX| for \verb|\mathbb{X}|), \verb|\mathcal| (\verb|\cX| for \verb|\mathcal{X}|), and \verb|\mathscr| (\verb|\sX| for \verb|\mathscr{X}|). E.g. \(\RR\) (\verb|\RR|), \(\cT\) (\verb|\cT|), \(\sK\) (\verb|\sK|). (Note that these shortcuts are not available for all letters.)
  
  \item A better looking mod: \(x \equiv y \mod 3\) (\verb|x \equiv y \mod 3|). 
\end{itemize}


\subsection{Math symbols by field}
\subsubsection*{Set Theory}
\begin{itemize}
  \item A better looking complement symbol: \(A^\complement\) (\verb|A^\complement|).
  \item A better empty set symbol: \(\emptyset\) (\verb|\emptyset|). 
  \item A cardinality command: \(\card{A}\) (\verb|\card{A}|). 
  \item A interior operator: \(\Int A\) (\verb|\Int A|). 
\end{itemize}

\subsubsection*{Probability}
\begin{itemize} 
  \item Operators: \(\Prob \E \var \Var \Cov\) (\verb|\Prob \E \var \Var \Cov|). 
\end{itemize}

\subsubsection*{Linear Algebra}
\begin{itemize} 
  \item Operators: \(\Id \Ker \tr \rank \RREF \almu \gemu \sign \Span\) \\
    (\verb|\Id \Ker \tr \rank \RREF \almu \gemu \sign \Span|). 
  \item Command for vectors: \(\vect{v}\) (\verb|\vect{v}|).
  \item Matrices: \[
    \bmat{1 & 2 \\ 3 & 4}, \pmat{1 & 2 \\ 3 & 4}, \vmat{1 & 2 \\ 3 & 4}
  \]
  (\verb|\bmat{1 & 2 \\ 3 & 4}, \pmat{1 & 2 \\ 3 & 4}, \vmat{1 & 2 \\ 3 & 4}|). 
\end{itemize}

\subsubsection*{Analysis}
\begin{itemize} 
  \item Differentiation operator: \(\dd x\) (\verb|\dd x|).
  \item Imaginary number: \(\I\) (\verb|\I|).
  \item Operators: \(\supp \epi \dist \Re \Im\) (\verb|\supp \epi \dist \Re \Im|).
\end{itemize}


\subsection{Miscellaneous}
\begin{itemize}
  \item Use \verb|\ds| as a shorthand for \verb|\displaystyle|. 
\end{itemize}


\section{Credits}
I have stolen a lot of stuff from \href{https://web.stanford.edu/~lindrew/}{Andrew Lin}'s package, \href{https://web.stanford.edu/~lindrew/lindrew.sty}{lindrew}, and \href{https://github.com/gillescastel}{Gilles Castel}'s \href{https://github.com/gillescastel/lecture-notes/blob/master/algebraic-topology/preamble.tex}{preamble file} for his \href{https://github.com/gillescastel/lecture-notes}{lecture notes}. 









\end{document}




