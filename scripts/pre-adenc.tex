\ProvidesPackage{adenc}[2025/05/06]


% ================================================
% ================ PACKAGE OPTIONS ===============
% ================================================
\newif\ifadenccolor \adenccolorfalse
\newif\ifadencplain \adencplaintrue
\newif\ifhideproofs \hideproofsfalse
\newif\ifhidemarkings \hidemarkingsfalse
\newif\ifworkingpaper \workingpaperfalse
\newif\iffancyqed \fancyqedfalse


% \usepackage[color]{adenc} to color theorem environments.
\DeclareOption{color}{\adenccolortrue}

% \usepackage[plain]{adenc} to use the default theorem environments: `definition`, `plain`, and `remark`.
\DeclareOption{plain}{\adencplaintrue}

% \usepackage[boxed]{adenc} to use the colored theorem environments
\DeclareOption{boxed}{\adencplainfalse}

% \usepackage[hideproofs]{adenc} to hide everything in the proof environment.
\DeclareOption{hideproofs}{\hideproofstrue}

% \usepackage[hidemarkings]{adenc} to hide all markings (`\markabove` and `\markbelow`; see below).
\DeclareOption{hidemarkings}{\hidemarkingstrue}

% \usepackage[workingpaper]{adenc} to get a larger page for the package todonotes
\DeclareOption{workingpaper}{\workingpapertrue}

% \usepackage[fancyqed]{adenc} to use the fancy qed (and other end-of-environment) symbols
\DeclareOption{fancyqed}{\fancyqedtrue}

\ProcessOptions*


% ================================================
% =============== STANDARD PACKAGES ==============
% ================================================
\RequirePackage{amsmath, amssymb, amsthm}
\RequirePackage{mathrsfs}

\RequirePackage{datetime}

\RequirePackage{float}
\RequirePackage{pdfpages}


\RequirePackage{tablefootnote}
\usepackage{pdfsync}
\synctex=1


% \RequirePackage[margin=1in]{geometry}

% increase pad 3cm to both left and right sides of the page for todonotes
\ifworkingpaper
  % \paperwidth=\dimexpr \paperwidth + 6cm\relax
  \paperwidth=\dimexpr \paperwidth + 3cm\relax
  % \oddsidemargin=\dimexpr\oddsidemargin + 3cm\relax
  \evensidemargin=\dimexpr\evensidemargin + 3cm\relax
  \marginparwidth=\dimexpr \marginparwidth + 3cm\relax

  \RequirePackage[firstpageonly=true]{draftwatermark}
  \SetWatermarkText{\sffamily%
    Draft \the\year/\twodigit\month/\twodigit\day%
  }
  % \today}
  \SetWatermarkColor{red!50}
  \SetWatermarkFontSize{20pt}
  \SetWatermarkAngle{0}
  \SetWatermarkVerCenter{0.1\paperheight}
  \SetWatermarkHorCenter{0.5\paperwidth}
  % \SetWatermarkText{DRAFT}
  % \SetWatermarkScale{1}
  % \usepackage[light,first,bottomafter]{draftcopy}
\fi


\RequirePackage[dvipsnames]{xcolor}
\RequirePackage{siunitx}
\RequirePackage{enumitem}


% ================================================
% ================ HYPERREF STYLES ===============
% ================================================
\RequirePackage{hyperref}
% citecolor: NavyBlue by default; blue if option `plain` used
\ifadencplain
  \newcommand{\citecolor}{blue}
\else
  \newcommand{\citecolor}{NavyBlue}
\fi

\hypersetup{
  bookmarksnumbered=true,
  colorlinks=true,
  linkcolor=\citecolor,
  citecolor=\citecolor,
  filecolor=\citecolor,
  menucolor=\citecolor,
  urlcolor=\citecolor,
  pdfnewwindow=true,
}


% ================================================
% ================== REFERENCES ==================
% ================================================
% \RequirePackage[natbib=true, backend=biber]{biblatex}
\RequirePackage{cleveref}
% \addbibresource{references.bib}


% ================================================
% ================ THEOREM STYLES ================
% ================================================
\RequirePackage{thmtools}
\RequirePackage[framemethod=TikZ]{mdframed}

% base style
\mdfdefinestyle{ThmBoxBaseStyle}{
  skipabove = .5em,
  skipbelow = .5em,
  linewidth = 1pt,
  innertopmargin = .25em,
  innerbottommargin = .5em,
  innerleftmargin = .75em,
  innerrightmargin = .75em,
  roundcorner = 3pt
}

% normal box: line skip after label
% (usage: `\getThmboxStyle{md style name}{theorem style name}{foreground color}{background color}`)
\newcommand{\getThmboxStyle}[4]{
  \mdfdefinestyle{#2}{
    style = ThmBoxBaseStyle, 
    linecolor = #3,
    backgroundcolor = #4,
    nobreak = true
  }
  \declaretheoremstyle[
    headfont = \bfseries\color{#3},
    notefont = \bfseries\color{#3},
    mdframed = {style = #2},
    headpunct = {.\\[0.5pt]},
    postheadspace = {0pt},
  ]{#1}
}

% compact box: no line skip after label
\newcommand{\getCompactThmboxStyle}[4]{
  \mdfdefinestyle{#2}{
    style = ThmBoxBaseStyle, 
    linecolor = #3,
    backgroundcolor = #4,
    nobreak = true,
    skipabove = 5pt,
    innertopmargin = 3pt,
    innerbottommargin = 5pt,
  }
  \declaretheoremstyle[
    headfont = \bfseries\color{#3},
    notefont = \bfseries\color{#3},
    mdframed = {style = #2},
    headpunct = {. },
    postheadspace = {0pt},
  ]{#1}
}

% colors for theorem environments
\definecolor{color1}{HTML}{052E66} % blue, thmbox
\definecolor{color2}{HTML}{8F0A00} % red, defbox
\definecolor{color3}{HTML}{2B4E2C} % green, exbox
\definecolor{color4}{HTML}{440793} % purple, notebox
\definecolor{color5}{HTML}{764506} % orange, probbox

% if no color passed, use black and white
\ifadenccolor
  \getThmboxStyle{thmbox}{mdthmbox}{color1}{color1!5}
  \getCompactThmboxStyle{defbox}{mddefbox}{color2}{color2!5}
  \getCompactThmboxStyle{exbox}{mdexbox}{color3}{color3!5}
  \getCompactThmboxStyle{notebox}{mdnotebox}{color4}{color4!5}
  \getCompactThmboxStyle{probbox}{mdprobbox}{color5}{color5!5}
\else
  \getThmboxStyle{thmbox}{mdthmbox}{black}{white}
  \getCompactThmboxStyle{defbox}{mddefbox}{black}{white}
  \getCompactThmboxStyle{exbox}{mdexbox}{black}{white}
  \getCompactThmboxStyle{notebox}{mdnotebox}{black}{white}
  \getCompactThmboxStyle{probbox}{mdprobbox}{black}{white}
\fi

% if option `plain` passed, use default styles: `plain`, `definition`, and `remark`
\ifadencplain
  \newcommand{\thmboxStyle}{plain}
  \newcommand{\defboxStyle}{definition}
  \newcommand{\exboxStyle}{remark}
  \newcommand{\noteboxStyle}{remark}
  \newcommand{\probboxStyle}{remark}
\else
  \newcommand{\thmboxStyle}{thmbox}
  \newcommand{\defboxStyle}{defbox}
  \newcommand{\exboxStyle}{exbox}
  \newcommand{\noteboxStyle}{notebox}
  \newcommand{\probboxStyle}{probbox}
\fi

\declaretheorem[style = \thmboxStyle, name = Theorem, numberwithin = section]{theorem}
\declaretheorem[style = \thmboxStyle, name = Theorem, numbered = no]{theorem*}
\declaretheorem[style = \thmboxStyle, name = Lemma, sibling = theorem]{lemma}
\declaretheorem[style = \thmboxStyle, name = Lemma, numbered = no]{lemma*}
\declaretheorem[style = \thmboxStyle, name = Proposition, sibling = theorem]{proposition}
\declaretheorem[style = \thmboxStyle, name = Proposition, numbered = no]{proposition*}
\declaretheorem[style = \thmboxStyle, name = Corollary, sibling = theorem]{corollary}
\declaretheorem[style = \thmboxStyle, name = Corollary, numbered = no]{corollary*}

\declaretheorem[style = \defboxStyle, name = Definition, sibling = theorem]{definition}
\declaretheorem[style = \defboxStyle, name = Definition, numbered = no]{definition*}

\declaretheorem[style = \exboxStyle, name = Example, sibling = theorem]{example}
\declaretheorem[style = \exboxStyle, name = Example, numbered = no]{example*}

\declaretheorem[style = \noteboxStyle, name = Remark, sibling=theorem]{remark}
\declaretheorem[style = \noteboxStyle, name = Remark, numbered=no]{remark*}
\declaretheorem[style = \noteboxStyle, name = Note, sibling=theorem]{note}
\declaretheorem[style = \noteboxStyle, name = Note, numbered=no]{note*}
\declaretheorem[style = \noteboxStyle, name = Fact, sibling = theorem]{fact}
\declaretheorem[style = \noteboxStyle, name = Fact, numbered = no]{fact*}
\declaretheorem[style = \noteboxStyle, name = Claim, sibling = theorem]{claim}
\declaretheorem[style = \noteboxStyle, name = Claim, numbered = no]{claim*}
\declaretheorem[style = \noteboxStyle, name = Notation, sibling = theorem]{notation}
\declaretheorem[style = \noteboxStyle, name = Notation, numbered = no]{notation*}
\declaretheorem[style = \noteboxStyle, name = Conjecture, sibling = theorem]{conjecture}
\declaretheorem[style = \noteboxStyle, name = Conjecture, numbered = no]{conjecture*}

\declaretheorem[style = \probboxStyle, name = Problem, sibling = theorem]{problem}
\declaretheorem[style = \probboxStyle, name = Problem, numbered = no]{problem*}
\declaretheorem[style = \probboxStyle, name = Question, sibling = theorem]{question}
\declaretheorem[style = \probboxStyle, name = Question, numbered = no]{question*}


% proof environment style:
\declaretheoremstyle[headfont=\bfseries, bodyfont=\normalfont, numbered=no, qed=\qedsymbol ]{thmproofbox}
\declaretheorem[numbered=no, style=thmproofbox, name=Proof]{replacementproof}
\renewenvironment{proof}[1][\proofname]{\begin{replacementproof}}{\end{replacementproof}}



% ================================================
% =================== GRAPHICS ===================
% ================================================
\RequirePackage{graphicx}
\RequirePackage{wrapfig}
\RequirePackage{subcaption}
\graphicspath{{./images/}}

\RequirePackage{tikz}
\RequirePackage{xy}


% ================================================
% ================ TODO COMMANDS =================
% ================================================
\RequirePackage{todonotes}
\RequirePackage{comment}
\newenvironment{nomarginbox}[2]{%
  \begin{mdframed}[
    innerleftmargin=0pt,
    innerrightmargin=0pt,
    innertopmargin=0pt,
    innerbottommargin=0pt,
    skipabove=0pt,
    skipbelow=0pt,
    backgroundcolor=#1,
    linecolor=#2,
    linewidth=1pt
  ]
}{%
  \end{mdframed}
}

\newcommand{\todocolor}{violet!75!red}

% \itodo and environments correction and lessimportant
\newenvironment{itodo}
  {\begin{nomarginbox}{green!3}{red}}
  {\end{nomarginbox}}

\newenvironment{correction}
  {\begin{nomarginbox}{red!1}{red!80!black}\color{red!80!black}}
  {\end{nomarginbox}}

\newenvironment{lessimportant}%
  {\begingroup\small\color{gray}}%
  {\endgroup}


% \markabove and \markbelow
\usetikzlibrary{arrows.meta}
\newcommand{\uppertip}{
  \tikz[anchor=base, baseline]{
    \draw[-{Triangle}, thick] (0,1ex) -- (0,1.1ex);
  }
}

\newcommand{\lowertip}{
  \tikz[anchor=base, baseline]{
    \draw[-{Triangle}, thick] (0,-0.5ex) -- (0,-0.6ex);
  }
}

% from: https://tex.stackexchange.com/questions/142242/robust-way-to-mark-draft-text
\RequirePackage{stackengine}
\setstackgap{L}{.5\baselineskip}
\newcommand\markabove[2]{{%
  \sffamily\bfseries\boldmath\color{\todocolor}%
  \hsmash{\uppertip}% 
  \smash{\toplap{#1}{\scriptsize#2}}%
}}
\newcommand\markbelow[2]{{%
  \sffamily\bfseries\boldmath\color{\todocolor}% 
  \hsmash{\lowertip}%
  \smash{\bottomlap{#1}{\scriptsize#2}}%
}}




% ================================================
% ============== HIDING ENVIRONMENTS =============
% ========= (Hiding markings and proofs) =========
% ================================================
\RequirePackage{environ}
\NewEnviron{hide}{}
% hide proofs if the `hideproofs` option is passed:
\ifhideproofs
  \let\proof\hide
  \let\endproof\endhide
\fi
% hide markings if the `hidemarkings` option is passed:
\ifhidemarkings
  \renewcommand{\markabove}[2]{}
  \renewcommand{\markbelow}[2]{}
\fi


% ================================================
% ================= FONTS/STYLES =================
% ================================================
% \usepackage{libertine}
% \usepackage{helvet} % PSNFSS Font, in every TeX distribution
% \renewcommand{\familydefault}{\sfdefault}

% \usepackage{newtxtext}
% \usepackage{newtxmath}

\RequirePackage{newtxtext}
\RequirePackage[varvw]{newtxmath}
\RequirePackage{courier}
% \usepackage{mathptmx}
% \usepackage{mathpazo}

\RequirePackage[cal=cm]{mathalpha}
\RequirePackage{framed}


% ================================================
% ================= MISCELLANEOUS ================
% ================================================
% blfootnote:
\newcommand\blfootnote[1]{%
  \begingroup
  \renewcommand\thefootnote{}\footnote{#1}%
  \addtocounter{footnote}{-1}%
  \endgroup
}

\newcommand{\ds}{\displaystyle}

\RequirePackage{silence} % for suppressing warnings
\WarningFilter{mdframed}{You have requested package}


% ================================================
% ===================== MATH =====================
% ================================================
\RequirePackage{amsfonts}
\RequirePackage{mathtools}

\newcommand\hmmax{0}
\newcommand\bmmax{0}
\RequirePackage{dsfont}
\usepackage{marvosym}
% \RequirePackage{bbm}
\RequirePackage{nth}
% ========== GENERAL MATH COMMANDS ==========
% vocab:
\newcommand{\vocab}[1]{\textbf{\color{black!90}\boldmath #1}}

% inverse trigs: \arcsec, \arccot, \arccsc
\DeclareMathOperator{\arcsec}{arcsec}
\DeclareMathOperator{\arccot}{arccot}
\DeclareMathOperator{\arccsc}{arccsc}

% contradiction
% \newcommand{\contradiction}{
%   \ensuremath{{\Rightarrow\mspace{-2mu}\Leftarrow}}
% }
\newcommand{\contradiction}{
  {\hbox{
    \setbox0=\hbox{\(\mkern-3mu{\times}\mkern-3mu\)}
    \setbox1=\hbox to0pt{\hss\copy0\hss}
    \copy0\raisebox{0.5\wd0}{\copy1}\raisebox{-0.5\wd0}{\box1}\box0}
  }
}

% better looking mod:
\renewcommand{\mod}[1]{\ \text{mod}\ #1}

% \abs and \norm
\DeclarePairedDelimiter\abs{\lvert}{\rvert}
\DeclarePairedDelimiter\norm{\lVert}{\rVert}
% floor and ceil
\DeclarePairedDelimiter\ceil{\lceil}{\rceil}
\DeclarePairedDelimiter\floor{\lfloor}{\rfloor}

% Swap the definition of \abs* and \norm*, so that \abs 
% and \norm resizes the size of the brackets, and the
% starred version does not.
\makeatletter
\let\oldabs\abs
\def\abs{\@ifstar{\oldabs}{\oldabs*}}
\let\oldnorm\norm
\def\norm{\@ifstar{\oldnorm}{\oldnorm*}}
\makeatother

% === shotcuts for mathbb, mathcal, and mathscr ===
\newcommand{\NN}{\mathbb{N}}
\newcommand{\ZZ}{\mathbb{Z}}
\newcommand{\QQ}{\mathbb{Q}}
\newcommand{\RR}{\mathbb{R}}
\newcommand{\CC}{\mathbb{C}}
\newcommand{\PP}{\mathbb{P}}
\newcommand{\FF}{\mathbb{F}}
\newcommand{\bS}{\mathbb{S}}
% === expanding \cX to \mathcal{X} for any capital letter X ===
\RequirePackage{xparse}
\RequirePackage{expl3}
\ExplSyntaxOn
\clist_map_inline:nn {A,B,C,D,E,F,G,H,I,J,K,L,M,N,O,P,Q,R,S,T,U,V,W,X,Y,Z}
{
  \cs_new:cpn { c#1 } { \mathcal{#1} }
}
\ExplSyntaxOff

% === mathscr ===
\newcommand{\sA}{\mathscr{A}}
\newcommand{\sB}{\mathscr{B}}
\newcommand{\sC}{\mathscr{C}}
\newcommand{\sD}{\mathscr{D}}
\newcommand{\sE}{\mathscr{E}}
\newcommand{\sS}{\mathscr{S}}

\newcommand{\sK}{\mathscr{K}}
\newcommand{\sL}{\mathscr{L}}
\newcommand{\sT}{\mathscr{T}}
\newcommand{\sU}{\mathscr{U}}
\newcommand{\sV}{\mathscr{V}}

% ==================== MATH, BY AREA ====================
\DeclareMathOperator*{\argmax}{arg\,max}
\DeclareMathOperator*{\argmin}{arg\,min}
% ========== Set Theory ==========
% complement: \complement
\let\Oldcomplement\complement
\renewcommand{\complement}[0]{\mathsf{c}}
% nicer emptyset
\let\oldemptyset\emptyset
\let\emptyset\varnothing
% cardinality: \card
\newcommand{\card}[1]{\left|#1\right|}
\newcommand{\dom}{\mathcal{D}}
% indicator function
\providecommand{\ind}{\ensuremath{\mathds{1}}}

% ========== Point Set Topology ==========
\DeclareMathOperator{\Int}{Int}

% ========== Probability ==========
\DeclareMathOperator{\var}{var}
\DeclareMathOperator{\Var}{Var}
\DeclareMathOperator{\sd}{sd}
\DeclareMathOperator{\se}{se}
\DeclareMathOperator{\Cov}{Cov}
\DeclareMathOperator{\Corr}{Corr}
\renewcommand{\Pr}{\mathbb{P}}
\renewcommand{\P}{\mathbb{P}}
\DeclareMathOperator{\E}{E}
\DeclareMathOperator{\MSE}{MSE}
\DeclareMathOperator{\Bias}{Bias}

\DeclareMathOperator{\disteq}{\stackrel{\mathscr{D}}{=}}
% \DeclareMathOperator{\distto}{\xrightarrow{\mathscr{D}}}
% \DeclareMathOperator{\longdistto}{\xrightarrow{\ \mathscr{D}\ }}
\DeclareMathOperator{\distto}{\to_{\mathrm{d}}}
\DeclareMathOperator{\longdistto}{\longrightarrow_{\mathrm{d}}}
\DeclareMathOperator{\probto}{\to_{\mathrm{p}}}
\DeclareMathOperator{\longprobto}{\longrightarrow_{\mathrm{p}}}
% \DeclareMathOperator{\longprobto}{\xrightarrow{\ p\ }}

\newcommand{\iidsim}{\overset{\mathrm{iid}}{\sim}}

\DeclareMathOperator{\Bernoulli}{Bernoulli}
\DeclareMathOperator{\Binomial}{Binomial}
\DeclareMathOperator{\Multinomial}{Multinomial}
\DeclareMathOperator{\Geometric}{Geometric}
\DeclareMathOperator{\Poisson}{Poisson}
\DeclareMathOperator{\Exponential}{Exponential}
\DeclareMathOperator{\DirichletDist}{Dirichlet}
\DeclareMathOperator{\GammaDist}{Gamma}
\DeclareMathOperator{\BetaDist}{Beta}
\DeclareMathOperator{\Normal}{\mathcal{N}}
\DeclareMathOperator{\Unif}{Uniform}
\DeclareMathOperator{\Uniform}{Uniform}

% ========== Linear Algebra ==========
\DeclareMathOperator{\Id}{Id}
\DeclareMathOperator{\tr}{tr}
\DeclareMathOperator{\rank}{rank}
\DeclareMathOperator{\RREF}{RREF}
\DeclareMathOperator{\almu}{almu}
\DeclareMathOperator{\gemu}{gemu}
\DeclareMathOperator{\sign}{sign}
\DeclareMathOperator{\Span}{span}

\newcommand*{\tran}{\intercal}

\newcommand\spanset[1]{\ensuremath\Span\left(#1\right)}
% % nicer \overline
% \let\oldoverline\overline
% \renewcommand{\overline}[1]{\mkern 1.5mu\oldoverline{\mkern-1.5mu#1\mkern-1.5mu}\mkern 1.5mu}

% % nicer \underline
% \let\oldunderline\underline
% \renewcommand{\underline}[1]{\mkern 1.5mu\oldunderline{\mkern-1.5mu#1\mkern-1.5mu}\mkern 1.5mu}

% nicer vector
\usepackage{bm}
\renewcommand{\vec}[1]{\boldsymbol{\mathbf{#1}}}
% \renewcommand{\vec}[1]{\mathbf{#1}}

% matrices
\newcommand{\bmat}[1]{\begin{bmatrix}#1\end{bmatrix}}
\newcommand{\vmat}[1]{\begin{vmatrix}#1\end{vmatrix}}
\newcommand{\pmat}[1]{\begin{pmatrix}#1\end{pmatrix}}

% ========== Analysis ==========
\renewcommand{\Re}{\operatorname{Re}}
\renewcommand{\Im}{\operatorname{Im}}
\newcommand*{\I}{\mathrm{i}}

\renewcommand{\d}{\mathrm{d}}
\newcommand{\dd}{\ \mathrm{d}}
\newcommand{\D}{\mathrm{D}}

\newcommand{\ran}{\mathcal{R}}
\renewcommand{\ker}{\mathcal{N}}

\DeclareMathOperator{\dist}{dist}
\DeclareMathOperator{\diam}{diam}

\DeclareMathOperator{\gr}{G}
\DeclareMathOperator{\graph}{G}
\DeclareMathOperator{\epi}{epi}

\DeclareMathOperator{\supp}{supp}

% ========== Spectral Theory ==========
\DeclareMathOperator{\res}{res}
\DeclareMathOperator{\spec}{spec}

% ========== Measure Theory ==========
\newcommand{\essran}{\mathrm{ess\ ran}\ }

% ========== ECONOMICS ==========
% ========== Econometrics ==========
\DeclareMathOperator{\SST}{SST}
\DeclareMathOperator{\SSE}{SSE}
\DeclareMathOperator{\SSR}{SSR}
\newcommand{\indep}{\perp \!\!\! \perp}




% ================================================
% ===================== CODE =====================
% ================================================
\usepackage{listings}
\definecolor{codegreen}{rgb}{0,0.6,0}
\definecolor{codegray}{rgb}{0.5,0.5,0.5}
\definecolor{codepurple}{rgb}{0.58,0,0.82}
\lstdefinestyle{lststyle}{
    commentstyle=\color{codegreen},
    keywordstyle=\color{magenta},
    numberstyle=\tiny\color{codegray},
    stringstyle=\color{codepurple},
    basicstyle=\ttfamily\footnotesize,
    breakatwhitespace=false,         
    breaklines=true,                 
    captionpos=b,                    
    keepspaces=true,                 
    numbers=left,                    
    numbersep=5pt,                  
    showspaces=false,                
    showstringspaces=false,
    showtabs=false,                  
    tabsize=2
}
\lstset{style=lststyle}


