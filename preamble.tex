% \documentclass{article}
\documentclass[10pt,letterpaper]{amsart}
% \documentclass[10pt,letterpaper,nocolor]{amsart}

\usepackage{amsmath, amssymb, amsthm}

\usepackage[letterpaper, margin=1in]{geometry}
\usepackage[usenames,dvipsnames]{xcolor}


\usepackage{xcolor}
\usepackage{hyperref}

\newcommand{\citecolor}{NavyBlue}
\hypersetup{
  bookmarksnumbered=true, 
  colorlinks=true, 
  linkcolor=\citecolor, 
  citecolor=\citecolor, 
  filecolor=\citecolor, 
  menucolor=\citecolor, 
  urlcolor=\citecolor, 
  pdfnewwindow=true, 
}
% \let\fullref\autoref
% %
% %  \autoref is very crude.  It uses counters to distinguish environments
% %  so that if say {lemma} uses the {theorem} counter, then autrorefs
% %  which should come out Lemma X.Y in fact come out Theorem X.Y.  To
% %  correct this give each its own counter eg:
% %                 \newtheorem{theorem}{Theorem}[section]
% %                 \newtheorem{lemma}{Lemma}[section]
% %  and then equate the counters by commands like:
% %                 \makeatletter
% %                   \let\c@lemma\c@theorem
% %                  \makeatother
% %
% %  To work correctly the environment name must have a corrresponding 
% %  \XXXautorefname defined.  The following command does the job:
% %
% \def\makeautorefname#1#2{\expandafter\def\csname#1autorefname\endcsname{#2}}
% %
% %  Some standard autorefnames.  If the environment name for an autoref 
% %  you need is not listed below, add a similar line to your TeX file:
% %  
% %\makeautorefname{equation}{Equation}%
% \def\equationautorefname~#1\null{(#1)\null}
% \makeautorefname{footnote}{footnote}%
% \makeautorefname{item}{item}%
% \makeautorefname{figure}{Figure}%
% \makeautorefname{table}{Table}%
% \makeautorefname{part}{Part}%
% \makeautorefname{appendix}{Appendix}%
% \makeautorefname{chapter}{Chapter}%
% \makeautorefname{section}{Section}%
% \makeautorefname{subsection}{Section}%
% \makeautorefname{subsubsection}{Section}%
% \makeautorefname{theorem}{Theorem}%
% \makeautorefname{corollary}{Corollary}%
% \makeautorefname{lemma}{Lemma}%
% \makeautorefname{proposition}{Proposition}%
% \makeautorefname{property}{Property}
% \makeautorefname{conjecture}{Conjecture}%
% \makeautorefname{definition}{Definition}%
% \makeautorefname{notation}{Notation}
% \makeautorefname{notations}{Notations}
% \makeautorefname{remark}{Remark}%
% \makeautorefname{question}{Question}%
% \makeautorefname{example}{Example}%
% \makeautorefname{axiom}{Axiom}%
% \makeautorefname{claim}{Claim}%
% \makeautorefname{assumption}{Assumption}%
% \makeautorefname{assumptions}{Assumptions}%
% \makeautorefname{construction}{Construction}%
% \makeautorefname{problem}{Problem}%
% \makeautorefname{warning}{Warning}%
% \makeautorefname{observation}{Observation}%
% \makeautorefname{convention}{Convention}%

% \makeautorefname{thm}{Theorem}
% \makeautorefname{cor}{Corollary}
% \makeautorefname{lem}{Lemma}
% \makeautorefname{prop}{Proposition}
% \makeautorefname{defn}{Definition}
% \makeautorefname{rem}{Remark}
% \makeautorefname{obs}{Observation}

% % %
% % %                  *** End of hyperref stuff ***




% ========== Theorem Styles ==========
\usepackage{thmtools}
\usepackage[framemethod=TikZ]{mdframed}




% \makeatletter

% \@ifclasswith{amsart}{nocolor}{
%   \declaretheoremstyle[headfont=\headfont, bodyfont=\normalfont]{thmgreenbox}
%   \declaretheoremstyle[headfont=\headfont, bodyfont=\normalfont]{thmredbox}
%   \declaretheoremstyle[headfont=\headfont, bodyfont=\normalfont]{thmbluebox}
%   \declaretheoremstyle[headfont=\headfont, bodyfont=\normalfont]{thmblueline}
%   % \declaretheoremstyle[headfont=\headfont, bodyfont=\normalfont, numbered=no, qed=\qedsymbol ]{thmproofbox}
%   % \declaretheorem[style=thmproofbox, name=Proof]{replacementproof}
%   % \renewenvironment{proof}[1][\proofname]{\begin{replacementproof}}{\end{replacementproof}}
% }{
%   \declaretheoremstyle[
%     headfont=\headfont\color{ForestGreen!70!black}, bodyfont=\normalfont,
%     mdframed={
%       linewidth=2pt,
%       rightline=false, topline=false, bottomline=false,
%       linecolor=ForestGreen, backgroundcolor=ForestGreen!3,
%     }
%   ]{thmgreenbox}

%   \declaretheoremstyle[
%     headfont=\headfont\color{NavyBlue!70!black}, bodyfont=\normalfont,
%     mdframed={
%       linewidth=2pt,
%       rightline=false, topline=false, bottomline=false,
%       linecolor=NavyBlue, backgroundcolor=NavyBlue!3,
%     }
%   ]{thmbluebox}

%   \declaretheoremstyle[
%     headfont=\headfont\color{NavyBlue!70!black}, bodyfont=\normalfont,
%     mdframed={
%       linewidth=2pt,
%       rightline=false, topline=false, bottomline=false,
%       linecolor=NavyBlue
%     }
%   ]{thmblueline}

%   \declaretheoremstyle[
%     headfont=\headfont\color{RawSienna!70!black}, bodyfont=\normalfont\itshape,
%     mdframed={
%       linewidth=2pt,
%       rightline=false, topline=false, bottomline=false,
%       linecolor=RawSienna, backgroundcolor=RawSienna!3,
%     }
%   ]{thmredbox}


%   % \declaretheoremstyle[
%   %     headfont=\headfont\color{NavyBlue!70!black}, bodyfont=\normalfont,
%   %     numbered=no,
%   %     mdframed={
%   %         linewidth=1.5pt,
%   %         rightline=false, topline=false, bottomline=false,
%   %         linecolor=NavyBlue, backgroundcolor=NavyBlue!1,
%   %     },
%   % ]{thmexplanationbox}

% }

% \makeatother

% ========== Theorem Styles ==========
% set proofs environment
% \declaretheoremstyle[
%   headfont=\bfseries, bodyfont=\normalfont,
%   numbered=no,
%   qed=\qedsymbol
% ]{thmproofbox}
% \declaretheorem[style=thmproofbox, name=Proof]{replacementproof}
% \renewenvironment{proof}[1][\proofname]{\vspace{-10pt}\begin{replacementproof}}{\end{replacementproof}}


% % theoremstyle{plain} --- default
% \usepackage{chngcntr}
% \counterwithin{equation}{section}

% % \swapnumbers

% \renewcommand{\theequation}{\arabic{equation}}


% \mdfsetup{skipabove=1em, skipbelow=0em, innertopmargin=5pt, innerbottommargin=6pt}
\mdfsetup{
  skipabove = 12pt, 
  skipbelow = 0pt, 
  linewidth = 1pt,
  innertopmargin = 2mm,
  innerbottommargin = 3.5mm,
  innerleftmargin = 3mm,
  innerrightmargin = 3mm, 
  roundcorner = 3pt
} % adjusts boundaries of boxes

\newcommand{\thmboxstyle}[4]{
  \mdfdefinestyle{#2}{
    linecolor = #3,
    backgroundcolor = #4,
    nobreak = true
  }
  \declaretheoremstyle[
    headfont = \bfseries\color{#3},
    % headfont = \sffamily\bfseries\color{#3},
    mdframed = {style = #2},
    headpunct = {\\[0.5pt]},
    postheadspace = {0pt},
  ]{#1}
}

% five different colors of boxes

% define colors:
\definecolor{color1}{HTML}{052E66} % blue
\definecolor{color2}{HTML}{8F0A00} % red
\definecolor{color3}{HTML}{2B4E2C} % green
\definecolor{color4}{HTML}{663352} % purple
\definecolor{color5}{HTML}{764506} % orange
\definecolor{color6}{HTML}{440793} % 
% \definecolor{color5}{HTML}{663352} % 
% \definecolor{color5}{HTML}{663352} % 

\thmboxstyle{thmbox}{mdthmbox}{color1}{color1!5}
\thmboxstyle{defbox}{mddefbox}{color2}{color2!5}
\thmboxstyle{exbox}{mdexbox}{color3}{color3!5}
\thmboxstyle{lembox}{mdlembox}{color4}{color4!5}
\thmboxstyle{notebox}{mdnotebox}{color5}{color5!5}
\thmboxstyle{problembox}{mdproblembox}{color6}{color6!5}

% \thmboxstyle{defbox}{mdredbox}{red}{orange!5}
% \thmboxstyle{defbox}{mdredbox}{RawSienna}{orange!5}
% \thmboxstyle{thmbox}{mdbluebox}{blue!90!red}{cyan!4}
% \thmboxstyle{thmbox}{mdbluebox}{NavyBlue!90!red}{cyan!4}
% \thmboxstyle{exbox}{mdgreenbox}{green!70!black}{teal!4}



\declaretheorem[style = thmbox, name = Theorem, numberwithin = section]{theorem}
\declaretheorem[style = thmbox, name = Theorem, numbered = no]{theorem*}
\declaretheorem[style = thmbox, name = Proposition, sibling = theorem]{proposition}
\declaretheorem[style = thmbox, name = Proposition, numbered = no]{proposition*}
\declaretheorem[style = thmbox, name = Corollary, sibling = theorem]{corollary}
\declaretheorem[style = thmbox, name = Corollary, numbered = no]{corollary*}
\declaretheorem[style = thmbox, name = Conjecture, sibling = theorem]{conjecture}
\declaretheorem[style = thmbox, name = Conjecture, numbered = no]{conjecture*}

\declaretheorem[style = lembox, name = Lemma, sibling = theorem]{lemma}
\declaretheorem[style = lembox, name = Lemma, numbered = no]{lemma*}

\declaretheorem[style = defbox, name = Definition, sibling = theorem]{definition}
\declaretheorem[style = defbox, name = Definition, numbered = no]{definition*}

\declaretheorem[style = exbox, name = Example, sibling = theorem]{example}
\declaretheorem[style = exbox, name = Example, numbered = no]{example*}

\declaretheorem[style = notebox, name = Remark, sibling=theorem]{remark}
\declaretheorem[style = notebox, name = Remark, numbered=no]{remark*}
\declaretheorem[style = notebox, name = Fact, sibling = theorem]{fact}
\declaretheorem[style = notebox, name = Fact, numbered = no]{fact*}
\declaretheorem[style = notebox, name = Claim, sibling = theorem]{claim}
\declaretheorem[style = notebox, name = Claim, numbered = no]{claim*}
\declaretheorem[style = notebox, name = Notation, sibling = theorem]{notation}
\declaretheorem[style = notebox, name = Notation, numbered = no]{notation*}

\declaretheorem[style = problembox, name = Problem, sibling = theorem]{problem}
\declaretheorem[style = problembox, name = Problem, numbered = no]{problem*}
\declaretheorem[style = problembox, name = Question, sibling = theorem]{question}
\declaretheorem[style = problembox, name = Question, numbered = no]{question*}









% ==================== Ref ====================
\usepackage{cleveref}
\usepackage{framed}



% ==================== Graphics ====================
\usepackage{graphicx}
\usepackage{wrapfig}
\usepackage{subcaption}
\graphicspath{{./images/}}



% ==================== MATH ====================
\usepackage{todonotes}
\usepackage{amsfonts}
\usepackage{bbm}
\usepackage{mathtools}
\usepackage{enumitem}
\usepackage{nth}

\newcommand{\ds}{\displaystyle}




% ==================== GENERAL MATH SYMBOLS ====================
% vocab:
\newcommand{\vocab}[1]{\textit{\textbf{\color{black!90}\boldmath #1}}}

% inverse trigs: \arcsec, \arccot, \arccsc
\DeclareMathOperator{\arcsec}{arcsec}
\DeclareMathOperator{\arccot}{arccot}
\DeclareMathOperator{\arccsc}{arccsc}

% contradiction
% \newcommand{\contradiction}{
%   \ensuremath{{\Rightarrow\mspace{-2mu}\Leftarrow}}
% }
\newcommand{\contradiction}{
  {\hbox{
    \setbox0=\hbox{\(\mkern-3mu{\times}\mkern-3mu\)}
    \setbox1=\hbox to0pt{\hss\copy0\hss}
    \copy0\raisebox{0.5\wd0}{\copy1}\raisebox{-0.5\wd0}{\box1}\box0}
  }
}

% \abs and \norm
\DeclarePairedDelimiter\abs{\lvert}{\rvert}
\DeclarePairedDelimiter\norm{\lVert}{\rVert}
% floor and ceil
\DeclarePairedDelimiter\ceil{\lceil}{\rceil}
\DeclarePairedDelimiter\floor{\lfloor}{\rfloor}

% Swap the definition of \abs* and \norm*, so that \abs
% and \norm resizes the size of the brackets, and the 
% starred version does not.
\makeatletter
\let\oldabs\abs
\def\abs{\@ifstar{\oldabs}{\oldabs*}}
\let\oldnorm\norm
\def\norm{\@ifstar{\oldnorm}{\oldnorm*}}
\makeatother


% === shotcuts for mathbb, mathcal, and mathscr === 
\newcommand{\NN}{\mathbb{N}}
\newcommand{\ZZ}{\mathbb{Z}}
\newcommand{\QQ}{\mathbb{Q}}
\newcommand{\RR}{\mathbb{R}}
\newcommand{\CC}{\mathbb{C}}
\newcommand{\PP}{\mathbb{P}}
\newcommand{\FF}{\mathbb{F}}
% expanding \cX to \mathcal{X} for any capital letter X
\usepackage{xparse}
\usepackage{expl3}
\ExplSyntaxOn
\clist_map_inline:nn {A,B,C,D,E,F,G,H,I,J,K,L,M,N,O,P,Q,R,S,T,U,V,W,X,Y,Z}
{
  \cs_new:cpn { c#1 } { \mathcal{#1} }
}
\ExplSyntaxOff
% === mathscr === 
\newcommand{\sA}{\mathscr{A}}
\newcommand{\sB}{\mathscr{B}}
\newcommand{\sC}{\mathscr{C}}
\newcommand{\sS}{\mathscr{S}}

\newcommand{\sk}{\mathscr{k}}
\newcommand{\sK}{\mathscr{K}}
\newcommand{\sL}{\mathscr{L}}
\newcommand{\sT}{\mathscr{T}}
\newcommand{\sU}{\mathscr{U}}
\newcommand{\sV}{\mathscr{V}}






% ==================== MATH, BY AREA ====================
% ========== Set Theory ==========
% complement: \complement
\let\Oldcomplement\complement
\renewcommand{\complement}[0]{\mathsf{c}}

% better looking mod:
\renewcommand{\mod}[1]{\ \text{mod}\ #1}

% nicer emptyset
\let\oldemptyset\emptyset
\let\emptyset\varnothing

% set: cardinality - \card
\newcommand{\card}[1]{\left|#1\right|}

% ========== Point Set Topology ==========
\DeclareMathOperator{\Int}{Int}

% ========== Probability ==========
\DeclareMathOperator{\var}{var}
\DeclareMathOperator{\Var}{Var}
\DeclareMathOperator{\Cov}{Cov}
% Probability: \Prob
% \newcommand{\Prob}[1]{\text{I\kern-0.15em P}[#1]}
\newcommand{\Prob}[1]{\mathbf{P}[#1]}

% ========== Linear Algebra ==========
\DeclareMathOperator{\Id}{Id}
\DeclareMathOperator{\Ker}{Ker}
\DeclareMathOperator{\tr}{tr}
\DeclareMathOperator{\rank}{rank}
\DeclareMathOperator{\RREF}{RREF}
\DeclareMathOperator{\almu}{almu}
\DeclareMathOperator{\gemu}{gemu}
\DeclareMathOperator{\sign}{sign}
\DeclareMathOperator{\Span}{span}
\newcommand\spanset[1]{\ensuremath\Span\left(#1\right)}
% bar: \overbar
\newcommand{\overbar}[1]{\mkern 1.5mu\overline{\mkern-1.5mu#1\mkern-1.5mu}\mkern 1.5mu}
% nicer underbar: \narrowunderbar
\newcommand{\narrowunderbar}[1]{\mkern 1.5mu\underline{\mkern-1.5mu#1\mkern-1.5mu}\mkern 1.5mu}
% nicer vector: \vect
% \newcommand{\vect}[1]{\boldsymbol{\narrowunderbar{#1}}}
\newcommand{\vect}[1]{\narrowunderbar{#1}}

% matrices
\newcommand{\bmat}[1]{\begin{bmatrix} #1 \end{bmatrix}}
\newcommand{\vmat}[1]{\begin{vmatrix} #1 \end{vmatrix}}
\newcommand{\pmat}[1]{\begin{pmatrix} #1 \end{pmatrix}}
 
% augmented matrix
% \newenvironment{amatrix}[1]{%
%   \left[\begin{array}{@{}*{#1}{c}|c@{}}
% }{%
%   \end{array}\right]
% }

% ========== Analysis ==========
\DeclareMathOperator{\dd}{d}
\DeclareMathOperator{\supp}{supp}
\DeclareMathOperator{\epi}{epi}
\DeclareMathOperator{\dist}{dist}

\renewcommand{\Re}{\operatorname{Re}}
\renewcommand{\Im}{\operatorname{Im}}
\newcommand*{\I}{\mathrm{i}}

% weak star convergence
\newcommand{\weakstarto}{\stackrel{\ast}{\rightharpoonup}}













% ========= Font/Style misc. =========
% \usepackage{eulervm}
% \usepackage{libertine}
% \usepackage{helvet} % PSNFSS Font, in every TeX distribution
% \renewcommand{\familydefault}{\sfdefault}

\usepackage{newtxtext}
\usepackage{newtxmath}
\usepackage[cal=cm]{mathalpha}

% \usepackage[small]{titlesec}
% conflicts with the use of \section when using amsart

% \renewcommand{\arraystretch}{1.25}





