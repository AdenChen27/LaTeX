\ProvidesPackage{adenc}[2024/08/14]


\newif\ifadenccolor \adenccolorfalse
\newif\ifadencplain \adencplainfalse
\newif\ifhideproofs \hideproofsfalse
\newif\ifhidemarkings \hidemarkingsfalse


\DeclareOption{color}{\adenccolortrue}

\DeclareOption{plain}{\adencplaintrue}

\DeclareOption{hideproofs}{\hideproofstrue}

\DeclareOption{hidemarkings}{\hidemarkingstrue}


\ProcessOptions*



\RequirePackage{amsmath, amssymb, amsthm}
\RequirePackage{mathrsfs}

\RequirePackage[letterpaper, margin=1in]{geometry}
\RequirePackage[dvipsnames]{xcolor}



\RequirePackage{hyperref}
\ifadencplain
  \newcommand{\citecolor}{blue}
\else
  \newcommand{\citecolor}{NavyBlue}
\fi

\hypersetup{
  bookmarksnumbered=true,
  colorlinks=true,
  linkcolor=\citecolor,
  citecolor=\citecolor,
  filecolor=\citecolor,
  menucolor=\citecolor,
  urlcolor=\citecolor,
  pdfnewwindow=true,
}

\RequirePackage{thmtools}
\RequirePackage[framemethod=TikZ]{mdframed}

\mdfdefinestyle{ThmBoxBaseStyle}{
  skipabove = 10pt,
  skipbelow = 0pt,
  linewidth = 1pt,
  innertopmargin = 5pt,
  innerbottommargin = 10pt,
  innerleftmargin = 10pt,
  innerrightmargin = 10pt,
  roundcorner = 3pt
}


\newcommand{\getThmboxStyle}[4]{
  \mdfdefinestyle{#2}{
    style = ThmBoxBaseStyle, 
    linecolor = #3,
    backgroundcolor = #4,
    nobreak = true
  }
  \declaretheoremstyle[
    headfont = \bfseries\color{#3},
    notefont = \bfseries\color{#3},
    mdframed = {style = #2},
    headpunct = {.\\[0.5pt]},
    postheadspace = {0pt},
  ]{#1}
}

\newcommand{\getCompactThmboxStyle}[4]{
  \mdfdefinestyle{#2}{
    style = ThmBoxBaseStyle, 
    linecolor = #3,
    backgroundcolor = #4,
    nobreak = true,
    skipabove = 5pt,
    innertopmargin = 3pt,
    innerbottommargin = 5pt,
  }
  \declaretheoremstyle[
    headfont = \bfseries\color{#3},
    notefont = \bfseries\color{#3},
    mdframed = {style = #2},
    headpunct = {. },
    postheadspace = {0pt},
  ]{#1}
}

\definecolor{color1}{HTML}{052E66} \definecolor{color2}{HTML}{8F0A00} \definecolor{color3}{HTML}{2B4E2C} \definecolor{color4}{HTML}{440793} \definecolor{color5}{HTML}{764506} 

\ifadenccolor
  \getThmboxStyle{thmbox}{mdthmbox}{color1}{color1!5}
  \getThmboxStyle{defbox}{mddefbox}{color2}{color2!5}
  \getCompactThmboxStyle{exbox}{mdexbox}{color3}{color3!5}
  \getCompactThmboxStyle{notebox}{mdnotebox}{color4}{color4!5}
  \getCompactThmboxStyle{probbox}{mdprobbox}{color5}{color5!5}
\else
  \getThmboxStyle{thmbox}{mdthmbox}{black}{white}
  \getThmboxStyle{defbox}{mddefbox}{black}{white}
  \getCompactThmboxStyle{exbox}{mdexbox}{black}{white}
  \getCompactThmboxStyle{notebox}{mdnotebox}{black}{white}
  \getCompactThmboxStyle{probbox}{mdprobbox}{black}{white}
\fi

\ifadencplain
  \newcommand{\thmboxStyle}{plain}
  \newcommand{\defboxStyle}{definition}
  \newcommand{\exboxStyle}{remark}
  \newcommand{\noteboxStyle}{remark}
  \newcommand{\probboxStyle}{remark}
\else
  \newcommand{\thmboxStyle}{thmbox}
  \newcommand{\defboxStyle}{defbox}
  \newcommand{\exboxStyle}{exbox}
  \newcommand{\noteboxStyle}{notebox}
  \newcommand{\probboxStyle}{probbox}
\fi

\declaretheorem[style = \thmboxStyle, name = Theorem, numberwithin = section]{theorem}
\declaretheorem[style = \thmboxStyle, name = Theorem, numbered = no]{theorem*}
\declaretheorem[style = \thmboxStyle, name = Lemma, sibling = theorem]{lemma}
\declaretheorem[style = \thmboxStyle, name = Lemma, numbered = no]{lemma*}
\declaretheorem[style = \thmboxStyle, name = Proposition, sibling = theorem]{proposition}
\declaretheorem[style = \thmboxStyle, name = Proposition, numbered = no]{proposition*}
\declaretheorem[style = \thmboxStyle, name = Corollary, sibling = theorem]{corollary}
\declaretheorem[style = \thmboxStyle, name = Corollary, numbered = no]{corollary*}

\declaretheorem[style = \defboxStyle, name = Definition, sibling = theorem]{definition}
\declaretheorem[style = \defboxStyle, name = Definition, numbered = no]{definition*}

\declaretheorem[style = \exboxStyle, name = Example, sibling = theorem]{example}
\declaretheorem[style = \exboxStyle, name = Example, numbered = no]{example*}

\declaretheorem[style = \noteboxStyle, name = Remark, sibling=theorem]{remark}
\declaretheorem[style = \noteboxStyle, name = Remark, numbered=no]{remark*}
\declaretheorem[style = \noteboxStyle, name = Note, sibling=theorem]{note}
\declaretheorem[style = \noteboxStyle, name = Note, numbered=no]{note*}
\declaretheorem[style = \noteboxStyle, name = Fact, sibling = theorem]{fact}
\declaretheorem[style = \noteboxStyle, name = Fact, numbered = no]{fact*}
\declaretheorem[style = \noteboxStyle, name = Claim, sibling = theorem]{claim}
\declaretheorem[style = \noteboxStyle, name = Claim, numbered = no]{claim*}
\declaretheorem[style = \noteboxStyle, name = Notation, sibling = theorem]{notation}
\declaretheorem[style = \noteboxStyle, name = Notation, numbered = no]{notation*}
\declaretheorem[style = \noteboxStyle, name = Conjecture, sibling = theorem]{conjecture}
\declaretheorem[style = \noteboxStyle, name = Conjecture, numbered = no]{conjecture*}

\declaretheorem[style = \probboxStyle, name = Problem, sibling = theorem]{problem}
\declaretheorem[style = \probboxStyle, name = Problem, numbered = no]{problem*}
\declaretheorem[style = \probboxStyle, name = Question, sibling = theorem]{question}
\declaretheorem[style = \probboxStyle, name = Question, numbered = no]{question*}


\declaretheoremstyle[headfont=\bfseries, bodyfont=\normalfont, numbered=no, qed=\qedsymbol ]{thmproofbox}
\declaretheorem[numbered=no, style=thmproofbox, name=Proof]{replacementproof}
\renewenvironment{proof}[1][\proofname]{\begin{replacementproof}}{\end{replacementproof}}
 



\RequirePackage{cleveref}


\RequirePackage{graphicx}
\RequirePackage{wrapfig}
\RequirePackage{subcaption}



\RequirePackage{amsfonts}
\RequirePackage{mathtools}
\RequirePackage{enumitem}
\RequirePackage{bbm}
\RequirePackage{nth}
\newcommand{\ds}{\displaystyle}



\newcommand{\vocab}[1]{\textbf{\color{black!90}\boldmath #1}}

\DeclareMathOperator{\arcsec}{arcsec}
\DeclareMathOperator{\arccot}{arccot}
\DeclareMathOperator{\arccsc}{arccsc}

\newcommand{\contradiction}{
  {\hbox{
    \setbox0=\hbox{\(\mkern-3mu{\times}\mkern-3mu\)}
    \setbox1=\hbox to0pt{\hss\copy0\hss}
    \copy0\raisebox{0.5\wd0}{\copy1}\raisebox{-0.5\wd0}{\box1}\box0}
  }
}

\renewcommand{\mod}[1]{\ \text{mod}\ #1}

\DeclarePairedDelimiter\abs{\lvert}{\rvert}
\DeclarePairedDelimiter\norm{\lVert}{\rVert}
\DeclarePairedDelimiter\ceil{\lceil}{\rceil}
\DeclarePairedDelimiter\floor{\lfloor}{\rfloor}

\makeatletter
\let\oldabs\abs
\def\abs{\@ifstar{\oldabs}{\oldabs*}}
\let\oldnorm\norm
\def\norm{\@ifstar{\oldnorm}{\oldnorm*}}
\makeatother


\newcommand{\NN}{\mathbb{N}}
\newcommand{\ZZ}{\mathbb{Z}}
\newcommand{\QQ}{\mathbb{Q}}
\newcommand{\RR}{\mathbb{R}}
\newcommand{\CC}{\mathbb{C}}
\newcommand{\PP}{\mathbb{P}}
\newcommand{\FF}{\mathbb{F}}
\newcommand{\bS}{\mathbb{S}}
\RequirePackage{xparse}
\RequirePackage{expl3}
\ExplSyntaxOn
\clist_map_inline:nn {A,B,C,D,E,F,G,H,I,J,K,L,M,N,O,P,Q,R,S,T,U,V,W,X,Y,Z}
{
  \cs_new:cpn { c#1 } { \mathcal{#1} }
}
\ExplSyntaxOff

\newcommand{\sA}{\mathscr{A}}
\newcommand{\sB}{\mathscr{B}}
\newcommand{\sC}{\mathscr{C}}
\newcommand{\sE}{\mathscr{E}}
\newcommand{\sS}{\mathscr{S}}

\newcommand{\sK}{\mathscr{K}}
\newcommand{\sL}{\mathscr{L}}
\newcommand{\sT}{\mathscr{T}}
\newcommand{\sU}{\mathscr{U}}
\newcommand{\sV}{\mathscr{V}}



\let\Oldcomplement\complement
\renewcommand{\complement}[0]{\mathsf{c}}
\let\oldemptyset\emptyset
\let\emptyset\varnothing
\newcommand{\card}[1]{\left|#1\right|}

\DeclareMathOperator{\Int}{Int}

\DeclareMathOperator{\var}{var}
\DeclareMathOperator{\Var}{Var}
\DeclareMathOperator{\Cov}{Cov}
\renewcommand{\Pr}{\mathbb{P}}
\DeclareMathOperator{\E}{\mathbb{E}}

\DeclareMathOperator{\Id}{Id}
\DeclareMathOperator{\ran}{ran}
\DeclareMathOperator{\Ker}{Ker}
\DeclareMathOperator{\tr}{tr}
\DeclareMathOperator{\rank}{rank}
\DeclareMathOperator{\RREF}{RREF}
\DeclareMathOperator{\almu}{almu}
\DeclareMathOperator{\gemu}{gemu}
\DeclareMathOperator{\sign}{sign}
\DeclareMathOperator{\Span}{span}

\newcommand\spanset[1]{\ensuremath\Span\left(#1\right)}




\renewcommand{\vec}[1]{\underline{#1}}

\newcommand{\bmat}[1]{\begin{bmatrix}#1\end{bmatrix}}
\newcommand{\vmat}[1]{\begin{vmatrix}#1\end{vmatrix}}
\newcommand{\pmat}[1]{\begin{pmatrix}#1\end{pmatrix}}

\renewcommand{\d}{\ \mathrm{d}}
\DeclareMathOperator{\supp}{supp}
\DeclareMathOperator{\epi}{epi}
\DeclareMathOperator{\dist}{dist}

\renewcommand{\Re}{\operatorname{Re}}
\renewcommand{\Im}{\operatorname{Im}}
\newcommand*{\I}{\mathrm{i}}


\RequirePackage{todonotes}

\newcommand{\todocolor}{violet!75!red}
\newcommand\itodo[1]{
  \begin{mdframed}[backgroundcolor=green!5, linewidth=1pt, linecolor=red]
    \color{\todocolor}#1
  \end{mdframed}
}

\RequirePackage{stackengine}
\setstackgap{L}{.5\baselineskip}
\newcommand\markabove[2]{{\sffamily\bfseries\color{\todocolor}\hsmash{\(\uparrow\)}\smash{\toplap{#1}{\scriptsize#2}}}}
\newcommand\markbelow[2]{{\sffamily\bfseries\color{\todocolor}\hsmash{\(\downarrow\)}\smash{\bottomlap{#1}{\scriptsize#2}}}}

\RequirePackage{environ}
\NewEnviron{hide}{}
\ifhideproofs
  \let\proof\hide
  \let\endproof\endhide
\fi
\ifhidemarkings
  \renewcommand{\markabove}[2]{}
  \renewcommand{\markbelow}[2]{}
\fi






\RequirePackage{framed}











