% base style
\mdfdefinestyle{ThmBoxBaseStyle}{
  skipabove = .5em,
  skipbelow = .5em,
  linewidth = 1pt,
  innertopmargin = .25em,
  innerbottommargin = .5em,
  innerleftmargin = .75em,
  innerrightmargin = .75em,
  roundcorner = 3pt
}

% normal box: line skip after label
% (usage: `\getThmboxStyle{md style name}{theorem style name}{foreground color}{background color}`)
\newcommand{\getThmboxStyle}[4]{
  \mdfdefinestyle{#2}{
    style = ThmBoxBaseStyle, 
    linecolor = #3,
    backgroundcolor = #4,
    nobreak = true
  }
  \declaretheoremstyle[
    headfont = \bfseries\color{#3},
    notefont = \bfseries\color{#3},
    mdframed = {style = #2},
    headpunct = {.\\[0.5pt]},
    postheadspace = {0pt},
  ]{#1}
}

% compact box: no line skip after label
\newcommand{\getCompactThmboxStyle}[4]{
  \mdfdefinestyle{#2}{
    style = ThmBoxBaseStyle, 
    linecolor = #3,
    backgroundcolor = #4,
    nobreak = true,
    skipabove = 5pt,
    innertopmargin = 3pt,
    innerbottommargin = 5pt,
  }
  \declaretheoremstyle[
    headfont = \bfseries\color{#3},
    notefont = \bfseries\color{#3},
    mdframed = {style = #2},
    headpunct = {. },
    postheadspace = {0pt},
  ]{#1}
}

% colors for theorem environments
\definecolor{color1}{HTML}{052E66} % blue, thmbox
\definecolor{color2}{HTML}{8F0A00} % red, defbox
\definecolor{color3}{HTML}{2B4E2C} % green, exbox
\definecolor{color4}{HTML}{440793} % purple, notebox
\definecolor{color5}{HTML}{764506} % orange, probbox

% if no color passed, use black and white
\ifadenccolor
  % \getThmboxStyle{thmbox}{mdthmbox}{color1}{color1!5}
  % \getThmboxStyle{defbox}{mddefbox}{color2}{color2!5}
  \getCompactThmboxStyle{thmbox}{mdthmbox}{color1}{color1!5}
  \getCompactThmboxStyle{defbox}{mddefbox}{color2}{color2!5}
  \getCompactThmboxStyle{exbox}{mdexbox}{color3}{color3!5}
  \getCompactThmboxStyle{notebox}{mdnotebox}{color4}{color4!5}
  \getCompactThmboxStyle{probbox}{mdprobbox}{color5}{color5!5}
\else
  % \getThmboxStyle{thmbox}{mdthmbox}{black}{white}
  % \getThmboxStyle{defbox}{mddefbox}{black}{white}
  \getCompactThmboxStyle{thmbox}{mdthmbox}{black}{white}
  \getCompactThmboxStyle{defbox}{mddefbox}{black}{white}
  \getCompactThmboxStyle{exbox}{mdexbox}{black}{white}
  \getCompactThmboxStyle{notebox}{mdnotebox}{black}{white}
  \getCompactThmboxStyle{probbox}{mdprobbox}{black}{white}
\fi

% if option `plain` passed, use default styles: `plain`, `definition`, and `remark`
\ifadencplain
  \newcommand{\thmboxStyle}{plain}
  \newcommand{\defboxStyle}{definition}
  \newcommand{\exboxStyle}{remark}
  \newcommand{\noteboxStyle}{remark}
  \newcommand{\probboxStyle}{remark}
\else
  \newcommand{\thmboxStyle}{thmbox}
  \newcommand{\defboxStyle}{defbox}
  \newcommand{\exboxStyle}{exbox}
  \newcommand{\noteboxStyle}{notebox}
  \newcommand{\probboxStyle}{probbox}
\fi

\declaretheorem[style = \thmboxStyle, name = Theorem, numberwithin = section]{theorem}
\declaretheorem[style = \thmboxStyle, name = Theorem, numbered = no]{theorem*}
\declaretheorem[style = \thmboxStyle, name = Lemma, sibling = theorem]{lemma}
\declaretheorem[style = \thmboxStyle, name = Lemma, numbered = no]{lemma*}
\declaretheorem[style = \thmboxStyle, name = Proposition, sibling = theorem]{proposition}
\declaretheorem[style = \thmboxStyle, name = Proposition, numbered = no]{proposition*}
\declaretheorem[style = \thmboxStyle, name = Corollary, sibling = theorem]{corollary}
\declaretheorem[style = \thmboxStyle, name = Corollary, numbered = no]{corollary*}

\declaretheorem[style = \defboxStyle, name = Definition, sibling = theorem]{definition}
\declaretheorem[style = \defboxStyle, name = Definition, numbered = no]{definition*}

\declaretheorem[style = \exboxStyle, name = Example, sibling = theorem]{example}
\declaretheorem[style = \exboxStyle, name = Example, numbered = no]{example*}

\declaretheorem[style = \noteboxStyle, name = Remark, sibling=theorem]{remark}
\declaretheorem[style = \noteboxStyle, name = Remark, numbered=no]{remark*}
\declaretheorem[style = \noteboxStyle, name = Note, sibling=theorem]{note}
\declaretheorem[style = \noteboxStyle, name = Note, numbered=no]{note*}
\declaretheorem[style = \noteboxStyle, name = Intuition, sibling=theorem]{intuition}
\declaretheorem[style = \noteboxStyle, name = Intuition, numbered=no]{intuition*}
\declaretheorem[style = \noteboxStyle, name = Properties, sibling=theorem]{properties}
\declaretheorem[style = \noteboxStyle, name = Properties, numbered=no]{properties*}
\declaretheorem[style = \noteboxStyle, name = Fact, sibling = theorem]{fact}
\declaretheorem[style = \noteboxStyle, name = Fact, numbered = no]{fact*}
\declaretheorem[style = \noteboxStyle, name = Claim, sibling = theorem]{claim}
\declaretheorem[style = \noteboxStyle, name = Claim, numbered = no]{claim*}
\declaretheorem[style = \noteboxStyle, name = Notation, sibling = theorem]{notation}
\declaretheorem[style = \noteboxStyle, name = Notation, numbered = no]{notation*}
\declaretheorem[style = \noteboxStyle, name = Conjecture, sibling = theorem]{conjecture}
\declaretheorem[style = \noteboxStyle, name = Conjecture, numbered = no]{conjecture*}

\declaretheorem[style = \probboxStyle, name = Problem, sibling = theorem]{problem}
\declaretheorem[style = \probboxStyle, name = Problem, numbered = no]{problem*}
\declaretheorem[style = \probboxStyle, name = Question, sibling = theorem]{question}
\declaretheorem[style = \probboxStyle, name = Question, numbered = no]{question*}


% proof environment style:
\declaretheoremstyle[headfont=\bfseries, bodyfont=\normalfont, numbered=no, qed=\qedsymbol ]{thmproofbox}
\declaretheorem[numbered=no, style=thmproofbox, name=Proof]{replacementproof}
\renewenvironment{proof}[1][\proofname]{\begin{replacementproof}}{\end{replacementproof}}
