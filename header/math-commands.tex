% ========== GENERAL MATH COMMANDS ==========
% vocab:
\newcommand{\vocab}[1]{\textbf{\color{black!90}\boldmath #1}}

% inverse trigs: \arcsec, \arccot, \arccsc
\DeclareMathOperator{\arcsec}{arcsec}
\DeclareMathOperator{\arccot}{arccot}
\DeclareMathOperator{\arccsc}{arccsc}

% contradiction
% \newcommand{\contradiction}{
%   \ensuremath{{\Rightarrow\mspace{-2mu}\Leftarrow}}
% }
\newcommand{\contradiction}{
  {\hbox{
    \setbox0=\hbox{\(\mkern-3mu{\times}\mkern-3mu\)}
    \setbox1=\hbox to0pt{\hss\copy0\hss}
    \copy0\raisebox{0.5\wd0}{\copy1}\raisebox{-0.5\wd0}{\box1}\box0}
  }
}

% better looking mod:
\renewcommand{\mod}[1]{\ \text{mod}\ #1}

% \abs and \norm
\DeclarePairedDelimiter\abs{\lvert}{\rvert}
\DeclarePairedDelimiter\norm{\lVert}{\rVert}
% floor and ceil
\DeclarePairedDelimiter\ceil{\lceil}{\rceil}
\DeclarePairedDelimiter\floor{\lfloor}{\rfloor}

% Swap the definition of \abs* and \norm*, so that \abs 
% and \norm resizes the size of the brackets, and the
% starred version does not.
\makeatletter
\let\oldabs\abs
\def\abs{\@ifstar{\oldabs}{\oldabs*}}
\let\oldnorm\norm
\def\norm{\@ifstar{\oldnorm}{\oldnorm*}}
\makeatother


% === shotcuts for mathbb, mathcal, and mathscr ===
\newcommand{\NN}{\mathbb{N}}
\newcommand{\ZZ}{\mathbb{Z}}
\newcommand{\QQ}{\mathbb{Q}}
\newcommand{\RR}{\mathbb{R}}
\newcommand{\CC}{\mathbb{C}}
\newcommand{\PP}{\mathbb{P}}
\newcommand{\FF}{\mathbb{F}}
\newcommand{\bS}{\mathbb{S}}
% === expanding \cX to \mathcal{X} for any capital letter X ===
\RequirePackage{xparse}
\RequirePackage{expl3}
\ExplSyntaxOn
\clist_map_inline:nn {A,B,C,D,E,F,G,H,I,J,K,L,M,N,O,P,Q,R,S,T,U,V,W,X,Y,Z}
{
  \cs_new:cpn { c#1 } { \mathcal{#1} }
}
\ExplSyntaxOff

% === mathscr ===
\newcommand{\sA}{\mathscr{A}}
\newcommand{\sB}{\mathscr{B}}
\newcommand{\sC}{\mathscr{C}}
\newcommand{\sD}{\mathscr{D}}
\newcommand{\sE}{\mathscr{E}}
\newcommand{\sS}{\mathscr{S}}

\newcommand{\sK}{\mathscr{K}}
\newcommand{\sL}{\mathscr{L}}
\newcommand{\sT}{\mathscr{T}}
\newcommand{\sU}{\mathscr{U}}
\newcommand{\sV}{\mathscr{V}}



% ==================== MATH, BY AREA ====================
% ========== Set Theory ==========
% complement: \complement
\let\Oldcomplement\complement
\renewcommand{\complement}[0]{\mathsf{c}}
% nicer emptyset
\let\oldemptyset\emptyset
\let\emptyset\varnothing
% cardinality: \card
\newcommand{\card}[1]{\left|#1\right|}
\newcommand{\dom}{\mathcal{D}}
% indicator function
\providecommand{\ind}{\ensuremath{\mathds{1}}}

% ========== Point Set Topology ==========
\DeclareMathOperator{\Int}{Int}

% ========== Probability ==========
\DeclareMathOperator{\var}{var}
\DeclareMathOperator{\Var}{Var}
\DeclareMathOperator{\Cov}{Cov}
\DeclareMathOperator{\Corr}{Corr}
\renewcommand{\Pr}{\mathbb{P}}
\renewcommand{\P}{\mathbb{P}}
\DeclareMathOperator{\E}{\mathbb{E}}
\DeclareMathOperator{\disteq}{\stackrel{\mathscr{D}}{=}}
\DeclareMathOperator{\distto}{\xrightarrow{\mathscr{D}}}
\DeclareMathOperator{\probto}{\xrightarrow{p}}
\newcommand{\iidsim}{\overset{\mathrm{iid}}{\sim}}

\DeclareMathOperator{\Bernoulli}{Bernoulli}
\DeclareMathOperator{\Bin}{Binomial}
\DeclareMathOperator{\Binomial}{Binomial}
\DeclareMathOperator{\Geometric}{Geometric}
\DeclareMathOperator{\Poisson}{Poisson}
\DeclareMathOperator{\Exponential}{Exponential}
\DeclareMathOperator{\GammaDist}{Gamma}
\DeclareMathOperator{\Normal}{N}
\DeclareMathOperator{\Unif}{Uniform}
\DeclareMathOperator{\Uniform}{Uniform}

% ========== Linear Algebra ==========
\DeclareMathOperator{\Id}{Id}
\DeclareMathOperator{\tr}{tr}
\DeclareMathOperator{\rank}{rank}
\DeclareMathOperator{\RREF}{RREF}
\DeclareMathOperator{\almu}{almu}
\DeclareMathOperator{\gemu}{gemu}
\DeclareMathOperator{\sign}{sign}
\DeclareMathOperator{\Span}{span}

\newcommand\spanset[1]{\ensuremath\Span\left(#1\right)}
% % nicer \overline
% \let\oldoverline\overline
% \renewcommand{\overline}[1]{\mkern 1.5mu\oldoverline{\mkern-1.5mu#1\mkern-1.5mu}\mkern 1.5mu}

% % nicer \underline
% \let\oldunderline\underline
% \renewcommand{\underline}[1]{\mkern 1.5mu\oldunderline{\mkern-1.5mu#1\mkern-1.5mu}\mkern 1.5mu}

% nicer vector
\renewcommand{\vec}[1]{\mathbf{#1}}


% matrices
\newcommand{\bmat}[1]{\begin{bmatrix}#1\end{bmatrix}}
\newcommand{\vmat}[1]{\begin{vmatrix}#1\end{vmatrix}}
\newcommand{\pmat}[1]{\begin{pmatrix}#1\end{pmatrix}}

% ========== Analysis ==========
\renewcommand{\Re}{\operatorname{Re}}
\renewcommand{\Im}{\operatorname{Im}}
\newcommand*{\I}{\mathrm{i}}

\renewcommand{\d}{\mathrm{d}}
\newcommand{\dd}{\ \mathrm{d}}
\newcommand{\D}{\mathrm{D}}

\newcommand{\ran}{\mathcal{R}}
\renewcommand{\ker}{\mathcal{N}}

\DeclareMathOperator{\dist}{dist}
\DeclareMathOperator{\diam}{diam}

\DeclareMathOperator{\gr}{G}
\DeclareMathOperator{\graph}{G}
\DeclareMathOperator{\epi}{epi}

\DeclareMathOperator{\supp}{supp}


% ========== Spectral Theory ==========
\DeclareMathOperator{\res}{res}
\DeclareMathOperator{\spec}{spec}

% ========== Measure Theory ==========
\newcommand{\essran}{\mathrm{ess\ ran}\ }




